\chapter{Motor Speed Calculations}
\label{chap:MSC}
This chapter describes the calculations used to synchronize the motors in the \Product{}. It assumes second order bang-coast-bang trajectories, which has up to three stages:
\begin{enumerate}
    \item In the first stage, the motor accelerates with maximum acceleration from its initial velocity $v_0$ to its cruising velocity $v_{cruise}$. This is the first bang.
    \item In the second stage, the motor maintains a constant velocity ($v_{cruise}$). This is the coast.
    \item In the third and final stage, the motor decelerates to its target velocity $v_{target}$. This is the second bang.
\end{enumerate}

\noindent The goal is to make $v_{cruise}$ as high as possible (as long as its lower than the maximum velocity $v_{max}$), to reach the target as quickly as possible. Note that, by changing $v_0$, $v_{target}$ or the total distance we need to travel, the total trajectory can be any combination of these three stages, as long as they remain in this order (e.g., not bang-bang-coast, but bang-bang or bang-coast are possible with certain parameters).

% Include the rest. Note: cannot use include due nested includes.
\section{Time Estimation}
\label{sec:timeEst}
This section elaborates on the calculation of the time estimation. This time estimation is calculated per motor from which the highest time estimation is used to synchronize all the motors.

To estimate the time needed to reach a target, two situations have to be considered, with $v_0$ the initial velocity, $v_{max}$ the maximum allowable velocity of that motor and $v_{target}$ the final velocity of that motor:
\begin{enumerate}
    \item $v_{max}$ is reached;
    \item $v_{max}$ is not reached.
\end{enumerate}

\noindent Whether or not $v_{max}$ can be reached can be checked with $t_{crit,0}$/$t_{crit,target}$ and $s_{crit}$, where $t_{crit,x}$ is the time required to reach $v_{max}$ from $v_x$ and $s_{crit}$ is the distance travelled when reaching $v_{max}$ from $v_0$ and immediately decelerating to $v_{target}$ (i.e., a bang-bang-trajectory). This type of trajectory is shown in \cref{fig: MSC_crit}, where the area under $v(t)$ is equal to $s_{crit}$. 

\begin{figure}[!hbt]
    \centering
    \includegraphics[width=0.7\textwidth]{figures/MotorSpeedCalculations/MSC_crit_trajectory.png}
    \caption{Critical trajectory when reaching maximum speed.}
    \label{fig: MSC_crit}
\end{figure}

\noindent Expressed in the given terms, $t_{crit,0}$/$t_{crit,target}$ can be found as shown in \cref{eq: tCrit}, and $s_{crit}$ can be found as shown in \cref{eq: sCrit}. 
\begin{align}
    t_{crit,x} &= \frac{\Delta v}{a_{max}} = \frac{v_{max}-v_x}{a_{max}} \Rightarrow t_{crit,0} = \frac{v_{max} - v_0}{a_{max}}, \; t_{crit,F} = \frac{v_{max} - v_F}{a_{max}} \label{eq: tCrit} \\
    s_{crit} &= \underbrace{\frac{1}{2} (v_{max}+v_0) \,  t_{crit,0}}_{\text{First bang}} + \underbrace{\frac{1}{2} (v_{max}+v_{target}) \, t_{crit,target}}_{\text{Second bang}}\nonumber \\
    & =\frac{v_{max}^2-v_0^2}{2 \, a_{max}}  + \frac{v_{max}^2-v_{target}^2}{2 \, a_{max}} \nonumber\\
    &= \frac{2 \, v_{max}^2-v_0^2 - v_{target}^2}{2 \, a_{max}} \label{eq: sCrit}
\end{align}

\noindent Then, it is very easy to decide which of the two situations the motor will be in, given a target distance $s_{target}$:
\medskip
\begin{algorithmic}
\If{$s_{target} \geq s_{crit}$} 
    \State $v_{max}$ reached
\Else
    \State $v_{max}$ not reached
\EndIf 
\end{algorithmic}
 \medskip
\noindent These situations require different calculations to get the required time to reach the target. These calculations when $v_{max}$ is reached are worked out in \cref{sub: MSC_speedreached}, and \cref{sub: MSC_speednotreached} shows the calculations for the other scenario.


\subsection{Situation 1: Maximum Speed Reached} \label{sub: MSC_speedreached}
\noindent In this situation, the motor reaches $v_{max}$ and thus follows a bang-coast-bang trajectory. The velocity of the motor over time in this situation is shown in \cref{fig: MSC_BCB}.
\begin{figure}[H]
    \centering
    \includegraphics[width=0.7\textwidth]{figures/MotorSpeedCalculations/MSC_BCB_trajectory.png}
    \caption{Bang-coast-bang trajectory when $v_{max}$ is reached.}
    \label{fig: MSC_BCB}
\end{figure}

\noindent Now, filling in $v_0$ and $v_{target}$ in \cref{eq: tCrit,eq: sCrit} to obtain $t_{xy}$ and $s_{xy}$, the time/distance when moving from point $x$ to $y$:

\begin{align*}
    t_{01} &= t_{crit,0} = \frac{v_{max} - v_0}{a_{max}} \\
    t_{2target} &= t_{crit,target} = \frac{v_{max} - v_{target}}{a_{max}} \\
    s_{01} + s_{2target} &= s_{crit} = \frac{2 \, v_{max}^2+v_0^2 + v_{target}^2}{2 \, a_{max}}
\end{align*}

\noindent Then, the total duration from start of finish can be found, where the final result is shown in \cref{eq: tFastBCB}. 
\begin{align}
    s_{12} = s_{target} - s_{crit}  &= s_{target} - \frac{2 \, v_{max}^2-v_0^2 - v_{target}^2}{2 \, a_{max}} \nonumber \\
    t_{12} = \frac{s_{12}}{v_{max}} &= \frac{s_{target} - \frac{2 \, v_{max}^2-v_0^2 - v_{target}^2}{2 \, a_{max}}}{v_{max}} = \frac{s_{target}}{v_{max}} - \frac{v_{max}}{a_{max}} + \frac{v_0^2 + v_{target}^2}{2\, v_{max} \, a_{max}} \nonumber \\
    t_{0target} = t_{01} + t_{12} + t_{2target} &= \frac{v_{max} - v_0}{a_{max}}+ \frac{s_{target}}{v_{max}} - \frac{v_{max}}{a_{max}} + \frac{v_0^2 + v_{target}^2}{2\, v_{max} \, a_{max}} + \frac{v_{max} - v_{target}}{a_{max}}\nonumber \\
    &= \frac{v_{max} - v_0 - v_{target}}{a_{max}}+ \frac{s_{target}}{v_{max}} + \frac{v_0^2 + v_{target}^2}{2\, v_{max} \, a_{max}} \label{eq: tFastBCB}
\end{align}

\newpage
\subsection{Situation 2: Maximum Speed Not Reached} \label{sub: MSC_speednotreached}
\noindent In this situation, the motor does not reach $v_{max}$ and thus follows a bang-bang trajectory. The velocity of the motor over time in this situation is shown in \cref{fig: MSC_BCB}.
\begin{figure}[H]
    \centering
    \includegraphics[width=0.7\textwidth]{figures/MotorSpeedCalculations/MSC_BB_trajectory.png}
    \caption{Bang-bang trajectory when $v_{max}$ is not reached.}
    \label{fig: MSC_BB}
\end{figure}

\noindent Because $v_{max}$ is not reached, its maximum reached speed is (currently) unknown. This speed, $v_{cruise}$, has to be chosen such that the desired distance is exactly traversed with the bang-bang trajectory. This can be done by expressing the distance as a function of $v_{cruise}$, which can be done by replacing all terms of $v_{max}$ in \cref{eq: tCrit,eq: sCrit} with $v_{cruise}$. The end result is shown in \cref{eq: tFastBB}. 
\begin{align}
    t_{01}  &= \frac{v_{cruise} - v_0}{a_{max}} \quad (\text{N.B.: }t_{12} = 0) \nonumber \\
    t_{2target}  &= \frac{v_{cruise} - v_{target}}{a_{max}} \nonumber \\
    s_{01} + s_{2target} = s_{target} &= \frac{2 \, v_{cruise}^2-v_0^2 - v_{target}^2}{2 \, a_{max}} \Rightarrow v_{cruise} = \sqrt{s_{target} \, a_{max} + \frac{1}{2}(v_0^2 + v_{target}^2)} \nonumber \\
t_{0target} = t_{01} + t_{2target} &= \frac{v_{cruise}-v_0}{a_{max}} + \frac{v_{cruise}-v_{target}}{a_{max}} = \frac{2 \, v_{cruise}-v_0 - v_{target}}{a_{max}}\nonumber \\
    &= \frac{2\,\sqrt{s_{target} \, a_{max} + \frac{1}{2}(v_0^2 + v_{target}^2)} -v_0 - v_{target} }{a_{max}} \label{eq: tFastBB}
\end{align}
\section{Scaling Factor}
\label{sec:scalingFactor}
This section describes the calculations for the scaling factor based on a time requirement. The time requirement is calculated from a time estimation, described in \cref{sec:timeEst}. In order to synchronize different motors, the following procedure is followed:

\begin{itemize}
    \item Calculate the duration of the movement of each motor to its target position.
    \item For all motors that arrive before the slowest motor, decrease their $v_{max}$ such that they arrive at the same time as the slowest motor.
\end{itemize}

\noindent The ratio by which $v_{max}$ is reduced is denoted with $\alpha$. In \cref{sec:timeEst}, essentially two types of trajectories are discussed: bang-coast-bang trajectories ($v_{max}$ was reached) and bang-bang trajectories ($v_{max}$ was not reached). A bang-coast-bang trajectory with a lower $v_{max}$ always results in another bang-coast-bang trajectory, whereas a bang-bang trajectory now also must result in a bang-coast-bang trajectory (otherwise nothing was altered). Thus, all new trajectories follow the trajectory and calculations from \cref{sub: MSC_speedreached} (except for the slowest motor, which may follow a bang-bang trajectory). 

\bigskip
\noindent All motors reach their (scaled) $v_{max}$ and the goal is to reach the target at $t_{req}$, where it must be that $t_{req} > t_{fastest}$\footnote{Note that the slowest motor is excluded from these calculations, as this motor might follow a trajectory. Its scaling factor is already known to be zero.} with $t_{fastest}$ being the fastest trajectory as derived in \cref{eq: tFastBCB}. Then, if $v_{max}$ is scaled with a factor $\alpha$, the time to the target is found as:
\begin{align*}
    t_{req} = t_{0target} &= \frac{\alpha \, v_{max} - v_0 - v_{target}}{a_{max}}+ \frac{s_{target}}{\alpha \, v_{max}} + \frac{v_0^2 + v_{target}^2}{2\, \alpha \, v_{max} \, a_{max}} \\
    \alpha \, t_{req} &= \frac{\alpha^2 \, v_{max}}{a_{max}} - \alpha \frac{v_0 + v_{target}}{a_{max}}+ \frac{s_{target}}{v_{max}} + \frac{v_0^2 + v_{target}^2}{2\, v_{max} \, a_{max}} \\
    0 &= \underbrace{\frac{v_{max}}{a_{max}}}_{a} \alpha^2 + \underbrace{\left(-\frac{v_0 + v_{target}}{a_{max}} - t_{req}\right)}_{b}\alpha + \underbrace{\frac{s_{target}}{v_{max}} + \frac{v_0^2 + v_{target}^2}{2\, v_{max} \, a_{max}}}_{c}
\end{align*}
\noindent This quadratic equation can easily be solved, yielding two solutions:
\begin{align*}
    \alpha &= \frac{-b \pm \sqrt{b^2 - 4 \,a\,c}}{2\,a} \\
    &= \frac{\frac{v_0 + v_{target}}{a_{max}} + t_{req} \pm \sqrt{\left(\frac{v_0 + v_{target}}{a_{max}} + t_{req}\right)^2 - 4 \left(\frac{s_{target}}{a_{max}} + \frac{v_0^2 + v_{target}^2}{2\,a_{max}^2}\right)}}{2 \frac{v_{max}}{a_{max}}} \\
    &= \frac{v_0 + v_{target}  + t_{req} \, a_{max} \pm \sqrt{\left({v_0 + v_{target}} + t_{req} \, a_{max} \right)^2 - 2 \left(2 \, s_{target} \, a_{max} + v_0^2 + v_{target}^2\right)}}{2 \, v_{max}} 
\end{align*}

\noindent For a given $t_{req}$, an increasing $s_{target}$ should result in an increasing $\alpha$. As the square root gets smaller, this must mean the only correct solution is where we subtract the square root\footnote{Note that the extra solution was added through the square in the beginning of this section.}. Thus, the correct equation for $\alpha$ is shown in \cref{eq: scalingFactor}.
\begin{equation}
    \alpha = \frac{v_0 + v_{target}  + t_{req} \, a_{max} - \sqrt{\left({v_0 + v_{target}} + t_{req} \, a_{max} \right)^2 - 2 \left(2 \, s_{target} \, a_{max} + v_0^2 + v_{target}^2\right)}}{2 \, v_{max}} \label{eq: scalingFactor}
\end{equation}



